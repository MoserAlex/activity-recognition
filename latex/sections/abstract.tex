% Abstract

This paper examines a set of acceleration data, with the goal to train several machine learning algorithms. Thereby these algorithms should be able to differentiate between different activities and movements. This study highlights, that it is possible to create a generalized training set, even though the data was collected under a more naturalistic setting. On the one hand, this means that it is not necessary to fixate an accelerometer at the test subjects body, but it is sufficient to loosely place it in a trouser pocket. on the other hand this also means that it is not necessary to create a personalized training set for each individual to reach a high accuracy in activity recognition.

In the beginning the data was collected with a web application, using the accelerometer of a smart phone. Multiple people performed several activities, which were then classified. Since problems during the classification process occurred, the decision was made to start a new attempt with an online available data set.

This study highlights the progress of both prototypes. In case of the first prototype, this paper aims to provide reasons and explanations as to why these problems emerged, and how to avoid them in future applications. Based on the second attempt, this paper explains the necessary steps to process the raw acceleration data. It lists extractable features and judges their suitability for activity recognition. Finally it also takes a look at 3 different machine learning algorithms and their applicability.