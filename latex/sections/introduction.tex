\section{Introduction}
\label{section:introduction}

Activity recognition is a heavily researched topic, since its potential has yet to be fully utilized. If applied correctly it can support doctors from various medical fields, athletes during their training, or simply offer quality of life improvements for everyone.

Since mobile devices are a part of our everyday life, almost everyone carries an easily accessible accelerometer with them. They are equipped with versatile and sophisticated hardware. Their sensors are not only able to track GPS, direction and acceleration, but also audio, video, light, temperature and much more. Most important for the topic of this study is the data associated to the movement of the owner. These sensors are easy to read and provide the necessary data to analyze the activities and behavior of the user.

So the question arises, whether it is still necessary to collect movement data under clinical circumstances. Is it already possible to immediately get started by just downloading an application without the need to train it first? Typically when examining a patient they need to go through a series of tests, before being able to give an assessment. This takes a lot of time which both the user and the doctor must be willing to invest. In case of using a mobile application for personal use this can not be expected from people under normal circumstances. They are most likely not willing to invest the needed time, unless it is due to illness or other compelling situations.

A mobile application measuring the users data on the fly can be a significant improvement for physiotherapists. Their patients can just use their phone to recognize and classify their daily activities, instead of running tests for hours. This way more insight into the everyday life of the patient is offered, which allows for a profound diagnosis of certain conditions, like bone resorption. A general monitoring of exercises, including the usefulness of their effect, grant the physiotherapist more suitable assumptions about further treatment steps. 

Movement analysis can also be used as a preemptive measure. Elderly people become weaker until they reach the state of frailty, making them more prone to bone fractures, disorders and diseases. Therefore recognizing a decline in physical activity or a change in gait in time can help geriatrists to take counter measures against frailty early on and as a result to delay this development. Usually these precautions are based on the viewpoint of the geriatrist and physical tests outside of the patients accustomed environment. Having a tool to analyze their habits and activities can help the geriatrist greatly in giving a thorough diagnosis.

Medicine and health care are not the only fields of application though. Customizing the behaviour of our phone based on our current activity can work as a quality of life improvement. For example, if people go for a run, their phone could automatically switch to a non-disturbance mode, automatically redirecting calls to their voice mail. Fitness applications, or the phone by itself, could further use the collected data to evaluate whether the user is living healthy or not. Depending on the analysis, feedback regarding exercise recommendations could be given. 

Many studies provide ideas and strategies on how to approach activity recognition. Others offer an analysis of how accurate various machine learning algorithm can predict activities, or which features are best to be used. During the research for this paper barely any studies were found that used data which was gathered in a naturalistic setting. Most studies had a clinical setup, with acceleration sensors fixed to certain body parts, or running on a treadmill for example. Therefore this paper aims to answer the following questions.


\subsection{Research Intend}
First of all, is it possible to accomplish high accuracy of activity recognition when using data which is not collected under supervision in a clinical environment? Or is it sufficient to use loosely fixated accelerometer, like a mobile phone in a trouser pocket? Second, can a high enough classification probability\footnote{The accuracy describing, how often the correct classification is chosen} be reached when the gathered data of a test person is compared to movement data of other training subjects? Is it possible to predict activities and behavior, when the data is not compared to a personalized set of training data, but a generalized set consisting of many training subjects?

In both cases the expectation is that the accuracy will be lower. The main question is by how much. If it is not by a significant margin, the result could still have its own merits, favoring a vastly lower time investment at the cost of prediction accuracy. If this kind of activity recognition is successful, patients would not need to train an machine, but could instead immediately start with the analysis. This way more data could be gathered, especially with the possible development of future applications which incorporate this technology.

This study examines the accuracy of activity recognition when using data gathered under natural circumstances. The data is analyzed whether it is possible to create a generalized data set, which can be used for activity classification. It will try to answer, if it is still necessary to build up a new personalized training set for every user. This study takes a look at various machine learning algorithms, which features are best suited for classification and which activities are easiest to be recognized and differentiated. During the experiment a generalized data set will be created with the aim to classify activities of test subjects who are not part of said data set.


\subsection{Definitions}
In order to understand detailed results of the experiment the main objective needs to be defined beforehand. So the understanding of activity recognition in the context of this paper shall be explained.

Activity recognition or activity classification refers to the analysis and prediction of movement behaviors and patterns. An activity is a continuous and repetitive form of movement, like walking, running or climbing stairs. Every change in acceleration, rotation, shock and others, caused by certain movements are descriptive attributes of an activity. These descriptive elements are called features and are used for activity classification.

Further the understanding of a generalized and a personalized data set will be outlined. A data set refers to the collected data from the accelerometer. It contains the raw acceleration data for certain activities. A personalized data set contains the acceleration data of one test subject for all activities he or she has performed. A generalized data set on the other hand contains the personalized data sets of multiple individuals, preferably with acceleration data for each activity of every single participant.
