% Kurzfassung

Diese Studie untersucht ein Datenset an Beschleunigungsdaten, um mit diesen unterschiedliche Lernalgorithmen zu trainieren. Diese sollen dadurch in der Lage sein, verschiedene Aktivitäten und Bewegungen voneinander unterscheiden zu können. Die Studie zeichnet sich vor allem dadurch aus, dass es möglich ist auch mit Daten ein Trainingsset zu erstellen, die unter natürlicheren Bedingungen als im Labor gesammelt wurden. Einerseits bedeutet dies, dass ein Accelerometer nicht unbedingt am Körper fixiert sein muss, sondern auch einfach lose in der Hosentasche platziert sein kann. Andererseits zeigt es auf, dass es nicht nötig ist, ein individuelles und personalisiertes Trainingsset für jede Testperson zu erstellen, um eine hohe Genauigkeit bei der Klassifizierung von Aktivitäten dieser Person zu erreichen.

Anfangs wurde versucht, die Daten über eine Webapplikation mit dem Smartphone aufzuzeichnen. Mehrere Personen führten verschiedene Aktivitäten aus, welche anschließend klassifiziert werden sollten. Da es dabei aber zu Problemen in der Auswertung kam, wurde ein zweiter Versuch gestartet, bei dem ein online zur Verfügung stehendes Datenset genutzt wurde.

Diese Studie beleuchtet dabei den Verlauf der Entwicklung beider Prototypen. Im Falle des ersten Versuches werden mögliche Gründe für die unzureichenden Ergebnisse genannt. Anhand des zweiten Versuchs wird erklärt, welche Schritte zu beachten sind, wenn man ein Datenset für einen Lernalgorithmus präparieren muss. Es werden verschiedene extrahierbare Features genannt und diese auf ihre Eignung zur Klassifizierung von Aktivitäten beurteilt. Weiters werden noch 3 verschiedene Lernalgorithmen untersucht und angewendet.