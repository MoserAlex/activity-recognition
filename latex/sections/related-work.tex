\section{Related Work}
\label{section:related}

During prior research it was conducted that the choice of the correct machine learning algorithm and methods for pre-processing data is very important and can yield results with significant differences from one another. Studies such as "Human Activity Recognition via an Accelerometer-Enabled-Smartphone Using Kernel Discriminant Analysis" by \textcite[]{khan2010human}, as well as the paper "Activity Recognition using Cell Phone Accelerometers" by \textcite[]{kwapisz2011activity} achieved an activity recognition rate of up to 90\%. Another paper underlining this result is "Activity Recognition on an Accelerometer Embedded Mobile Phone with Varying Positions and Orientations" by \textcite[]{sun2010activity}. While all three delivered solid results, \textcite[]{kwapisz2011activity} fails to identify climbing stairs. The method of \textcite[]{khan2010human} using an \gls{ann} appears slightly superior to the \gls{svm} \textcite[]{sun2010activity} uses in terms of accuracy. However, an \gls{ann} and a \gls{svm} are both solid classification algorithms to choose for activity recognition in combination with the correct setup and enough training material.

Another noteworthy paper is "A Feature Extraction Method for Realtime Human Activity Recognition on Cell Phones" by \textcite[]{khan2011feature}. The researches from respectively the North South University in Dhaka (Bangladesh), the Marquette University in Milwaukee (USA) and the University of Wisconsin-Milwaukee in Milwaukee (USA) used a standard acceleration-based activity recognition data set, called \gls{scut-naa}, for extracting features from acceleration sensor signals. This allowed them to identify human activities, with which they hoped to contribute a novel linear-time method. The study revealed, that personalized training and testing data appears to be more useful than generalized data sets. Both of their methods brought them to the same conclusion. In this sense whether a \gls{svm} classifier with a linear kernel or with a \gls{rbf} kernel is applied, the level of accuracy remains the same.

The detailed researches ensured a knowledge base which helped to decide which technology is needed, to define the usefulness of the experiment itself as well as to estimate the expected outcome. The topics relevance in combination with the current technological developments ensure a positive outlook for the usage of activity recognition. From the throughout researched theoretical standpoint the practical part could be well structured. 


\subsection{Data Collection}
First of all the acceleration data from various test subjects needs to be gathered. A sophisticated data set is important, since it is the foundation of any test environment. \textcite[]{khan2011feature} for example used the \gls{scut-naa} data set. \textcite[]{xue2010naturalistic} created the data set by placing multiple sensors on the persons body and collecting the acceleration data via Bluetooth.

Alternatively a new set of data can be created, if the research team has new ideas on how to collect the data. In their study \textcite[]{kwapisz2011activity} used the accelerometer of a mobile phone to track the movement and acceleration of their test subjects.

Many studies performed the data collection on their own, but many also used preexisting data to save time and focus on their research question. If the way the data is collected is not important, then it is perfectly fine to use one that has been created by someone else. That is the reason why the \gls{scut-naa} data set is used in this study as well.

The previously mentioned studies collected various different activities. The most common of them were running, walking, climbing up or down the stairs, and resting. A test group was asked to perform these activities to collect the data and create personalized feature sets. Afterwards the data was analyzed and collectively evaluated. That way personalized sets of data for each individual were accessed, as well as an excessive set with the combined data was generated.

In the second part of the experiment the test group performed similar tasks as in the first part. The combined and personalized feature set were used to predict the activities of the test person. Additionally there were people in the second part of this test, which were not part of the learning process. This allowed for a fresh look at how well activities of people, who were not part of the first test group, can be recognized. To mimic this effect, this study will not train the classification algorithm with every test subject, but instead leave a few for testing.


\subsection{Data Processing}
There are different methods of reading and interpreting the signal from the accelerometer. For example the tilt of the device can be ignored, if you are only interested in the acceleration relative to its origin \autocite[]{kwapisz2011activity}. On the other hand, the tilt can also be used to estimate where the device is placed, if it is in the users hand, or in the users pocket \autocite[]{khan2011feature}. Another variable is the time window of how much data you should combine into one sample. While a 10-second segment might be sufficient to recognize long and repetitive movements like running or walking \autocite[]{kwapisz2011activity}, shorter activities might not be recognized, like climbing stairs.

After the pre-processing is completed, various features are calculated. \textcite[]{khan2010human} used \gls{ar} coefficients and the \gls{sma} to create feature vectors. Others like \textcite[]{kwapisz2011activity} used the average acceleration on each of the three axis, the standard deviation, the time between peaks and other to describe the characteristics of a movement. A collection of these feature vectors are called feature set. These are used by the aforementioned studies to train their algorithms and to test them.

\subsection{Machine Learning Algorithms}
The interpretation of the mobile data was done on a separated computer, which is powerful enough to handle performance intensive classification algorithms. A machine learning algorithm will be fed with the data in order to learn to distinguish between activities. The different studies found during the research process used a variety of different classification algorithms, including a decision tree, \gls{knn}, \gls{svm} and \gls{ann}. They also describe, how they configured them and how to avoid certain pitfalls.
